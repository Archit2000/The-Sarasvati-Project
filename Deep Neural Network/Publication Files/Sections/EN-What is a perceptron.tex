\section{What is a perceptron?}

    A perceptron is a fundamental building block in the world of deep learning! It's a simple yet powerful model that acts as a single artificial neuron within a neural network. Here's how it works:
    Imagine a perceptron as a gatekeeper. It takes in a bunch of information (represented by numbers) as its input, like the features of an image or the words in a sentence. Inside, it performs a calculation on this information using weights and a bias. Think of the weights as adjustable knobs that control how much each piece of information affects the decision. The bias is like a threshold that the combined weighted information needs to cross to pass through the gate.
    The key output of a perceptron is a binary classification: either 0 or 1. Based on its calculation, the perceptron decides whether the input belongs to one class or the other. For example, in image classification, it might decide if an image contains a cat or not.
    Here are some key points about perceptrons:
        \begin{itemize}
            \item \textbf{They are linear classifiers:} This means they can only draw straight lines to separate data points into different classes. This limits their ability to handle complex data patterns.
            \item \textbf{They are the foundation of neural networks:} Multi-layer perceptrons, which are networks of interconnected perceptrons, can handle more complex data by combining the decisions of multiple layers.
            \item \textbf{They are simple and easy to understand:} This makes them a great starting point for learning about deep learning and how neural networks work.
        \end{itemize}

