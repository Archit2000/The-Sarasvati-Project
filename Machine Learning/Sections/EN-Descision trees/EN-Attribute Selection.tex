\section{Attribute Selection}

    \subsection{Entropy and Information Gain}
        Here's a simple explanation of entropy in a decision tree, using a fruit basket analogy:

        Imagine a basket full of mixed fruits:
        \begin{enumerate}
            \item If it's all apples, there's no surprise when you pick one—it's always an apple. This is like a node with low entropy (more purity).
            \item If it's half apples and half oranges, there's more uncertainty—you might get either fruit. This is like a node with higher entropy (less purity).
        \end{enumerate}
        Entropy is the measure of uncertainty of a random variable, it characterizes the impurity of an arbitrary collection of examples. The higher the entropy more the information content. Entropy can be though of as a measure of disorder or impurity in a node. 
        Information Gain  = $Entropy_{Parent} - Entropy_{Children}$
        Decision trees use entropy to make choices:
        \begin{enumerate}
            \item They try to split nodes into groups that are more "pure," like sorting the fruits into separate baskets.
            \item A node with high entropy (mixed fruits) is a good candidate for splitting, as it can lead to more certainty in the branches.
        \end{enumerate}
        When we use a node in a decision tree to partition the training instances into smaller subsets the entropy changes. Information gain is a measure of this change in entropy.
        \begin{itemize}
            \item Suppose S is a set of instances,
            \item A is an attribute
            \item $S_v$ is the subset of S
            \item v represents an individual value that the attribute A can take and Values (A) is the set of all possible values of A, then
            \item Refer Formula Sheet.
        \end{itemize}
        
        \subsection{Gini Index}
        The Gini Index is a measure of the inequality or impurity of a distribution, commonly used in decision trees and other machine learning algorithms. It ranges from 0 to 1, where 0 represents perfect equality (all values are the same) and 1 represents perfect inequality (all values are different).
        In other words, Gini Index is a metric to measure how often a randomly chosen element would be incorrectly identified.
        It means an attribute with a lower Gini index should be preferred.
        Some additional features and characteristics of the Gini Index are:
        \begin{itemize}
            \item It is calculated by summing the squared probabilities of each outcome in a distribution and subtracting the result from 1.
            \item A lower Gini Index indicates a more homogeneous or pure distribution, while a higher Gini Index indicates a more heterogeneous or impure distribution.
            \item In decision trees, the Gini Index is used to evaluate the quality of a split by measuring the difference between the impurity of the parent node and the weighted impurity of the child nodes.
            \item Compared to other impurity measures like entropy, the Gini Index is faster to compute and more sensitive to changes in class probabilities.
            \item One disadvantage of the Gini Index is that it tends to favor splits that create equally sized child nodes, even if they are not optimal for classification accuracy.
            \item In practice, the choice between using the Gini Index or other impurity measures depends on the specific problem and dataset, and often requires experimentation and tuning.
        \end{itemize}

    \subsection{Difference between Gini Index and Entropy}
        \begin{center}
            \begin{tabularx}{0.8\textwidth} { 
                | >{\raggedright\arraybackslash}X 
                | >{\raggedright\arraybackslash}X 
                | >{\raggedright\arraybackslash}X | } 
            \hline
            Gini Index & Entropy \\ 
            \hline
            It is the probability of misclassifying a randomly chosen element in a set.	& While entropy measures the amount of uncertainty or randomness in a set.\\
            \hline
            The range of the Gini index is [0, 1], where 0 indicates perfect purity and 1 indicates maximum impurity. & The range of entropy is [0, log(c)], where c is the number of classes.\\
            \hline
            Gini index is a linear measure.	& Entropy is a logarithmic measure. \\ 
            \hline
            It can be interpreted as the expected error rate in a classifier. & It can be interpreted as the average amount of information needed to specify the class of an instance. \\
            \hline
            It is sensitive to the distribution of classes in a set. & It is sensitive to the number of classes. \\
            \hline
            It is less robust than entropy.	& It is more robust than Gini index.\\
            \hline
            It is sensitive. & It is comparatively less sensitive.\\
            \hline 
            It has a bias toward selecting splits that result in a more balanced distribution of classes. & It has a bias toward selecting splits that result in a higher reduction of uncertainty.\\
            \hline
            Gini index is typically used in CART (Classification and Regression Trees) algorithms & Entropy is typically used in ID3 and C4.5 algorithms. \\
            \hline
            \end{tabularx}
        \end{center}
