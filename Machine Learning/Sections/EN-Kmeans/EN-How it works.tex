\section{How k-means clustering works?}
    We are given a data set of items, with certain features, and values for these features like a vector. The task is to categorize those items into groups. To achieve this, we will use the K-means algorithm, an unsupervised learning algorithm. 'K' in the name of the algorithm represents the number of groups/clusters we want to classify our items into.

    It will help if you think of items as points in an n-dimensional space. The algorithm will categorize the items into k groups or clusters of similarity. To calculate that similarity, we will use the Euclidean distance as a measurement.

    The algorithm works as follows:  
    \begin{enumerate}
        \item First, we randomly initialize k points, called means or cluster centroids.
        \item We categorize each item to its closest mean, and we update the mean's coordinates, which are the averages of the items categorized in that cluster so far.
        \item We repeat the process for a given number of iterations and at the end, we have our clusters.
    \end{enumerate}
    The 'points' mentioned above are called means because they are the mean values of the items categorized in them. To initialize these means, we have a lot of options. An intuitive method is to initialize the means at random items in the data set. Another method is to initialize the means at random values between the boundaries of the data set (if for a feature x, the items have values in [0,3], we will initialize the means with values for x at [0,3]).

    The above algorithm in pseudocode is as follows:
    \begin{itemize}
        \item Initialize k means with random values
        \item For a given number of iterations:
            \begin{itemize}
                \item Iterate through items:
                    \begin{itemize}
                        \item Find the mean closest to the item by calculating the euclidean distance of the item with each of the means
                        \item Assign item to mean
                        \item Update mean by shifting it to the average of the items in that cluster
                    \end{itemize}
            \end{itemize}
    \end{itemize}