\section{Advantages and Disadvantages}
    \subsection{Advantages}
        \begin{itemize}
            \item It is very easy to understand and implement.
            \item If we have large number of variables then, K-means would be faster than Hierarchical clustering.
            \item On re-computation of centroids, an instance can change the cluster.
            \item Tighter clusters are formed with K-means as compared to Hierarchical clustering.
        \end{itemize}
    \subsection{Disadvantages}
        \begin{itemize}
            \item It is a bit difficult to predict the number of clusters i.e. the value of k.
            \item Output is strongly impacted by initial inputs like number of clusters (value of k)
            \item Order of data will have strong impact on the final output.
            \item It is very sensitive to rescaling. If we will rescale our data by means of normalization or standardization, then the output will completely change.
            \item It is not good in doing clustering job if the clusters have a complicated geometric shape.
        \end{itemize}