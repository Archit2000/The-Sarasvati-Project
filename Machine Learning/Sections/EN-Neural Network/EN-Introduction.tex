\section{Introduction}
    A neural network is a method in artificial intelligence that teaches computers to process data in a way that is inspired by the human brain. It is a type of machine learning process, called deep learning, that uses interconnected nodes or neurons in a layered structure that resembles the human brain.

\section{Types of Neural Networks}
    There are seven types of neural networks that can be used.
    \begin{itemize}
        \item Multilayer Perceptron (MLP): 
            \begin{itemize}
                \item A type of feedforward neural network with three or more layers, including an input layer, one or more hidden layers, and an output layer. It uses nonlinear activation functions.
            \end{itemize}
        \item Convolutional Neural Network (CNN): 
            \begin{itemize}
                \item A neural network that is designed to process input data that has a grid-like structure, such as an image. It uses convolutional layers and pooling layers to extract features from the input data.
            \end{itemize}
        \item Recursive Neural Network (RNN): 
            \begin{itemize}
                \item A neural network that can operate on input sequences of variable length, such as text. It uses weights to make structured predictions.
            \end{itemize}
        \item Recurrent Neural Network (RNN): 
            \begin{itemize}
                \item A type of neural network that makes connections between the neurons in a directed cycle, allowing it to process sequential data.
            \end{itemize}
        \item Long Short-Term Memory (LSTM): 
            \begin{itemize}
                \item A type of RNN that is designed to overcome the vanishing gradient problem in training RNNs. It uses memory cells and gates to selectively read, write, and erase information.
            \end{itemize}
        \item Sequence-to-Sequence (Seq2Seq): 
            \begin{itemize}
                \item A type of neural network that uses two RNNs to map input sequences to output sequences, such as translating one language to another.
            \end{itemize}
        \item Shallow Neural Network: 
            \begin{itemize}
                \item A neural network with only one hidden layer, often used for simpler tasks or as a building block for larger networks.
            \end{itemize}
    \end{itemize}