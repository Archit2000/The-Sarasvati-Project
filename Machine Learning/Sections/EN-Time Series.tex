\chapterimage{orange2.jpg} % Chapter heading image
\chapterspaceabove{6.75cm} % Whitespace from the top of the page to the chapter title on chapter pages
\chapterspacebelow{7.25cm} % Amount of vertical whitespace from the top margin to the start of the text on chapter pages

%------------------------------------------------

\chapter{Time Series Data}
    \section{Introduction}
        Time series data is a collection of observations recorded over time in a specific order. Each observation contains a value at a particular point in time, forming a sequence that captures how a variable changes over time.

        Here are some key characteristics of time series data:
        \begin{itemize}
            \item Ordered: The data points are ordered chronologically, emphasizing the temporal relationship between them.
            \item Dependent: Data points at different time points can be related and influence each other.
            \item Dynamic: The values can change over time, exhibiting trends, patterns, and seasonality.
            \item Multi-dimensional: Time series data can have multiple dimensions, with each point recording multiple attributes simultaneously.
        \end{itemize}
    
    \section{Exploratory Data Analysis}
        \subsection{Guide}
                Exploratory data analysis (EDA) is crucial for understanding your time series data before building a model. Here's a guide to conducting effective time series EDA:
                \begin{itemize}
                    \item Import Libraries:
                        \begin{itemize}    
                            \item Start by importing necessary libraries like Pandas, NumPy, Matplotlib, and Seaborn for data manipulation, analysis, and visualization.
                        \end{itemize}
                    \item Load and Inspect Data:
                        \begin{itemize}
                            \item Load your time series data into a DataFrame.
                            \item Check for missing values, data types, and basic summary statistics like mean, standard deviation, etc.
                            \item Investigate the time index format and ensure its correctness.
                        \end{itemize}
                    \item Visualize Trends and Seasonality:
                        \begin{itemize}
                            \item Plot the time series data using line charts or time series plots to observe overall trends, patterns, and potential seasonality.
                            \item Try different time granularities (daily, weekly, monthly) to identify trends at different scales.
                            \item Visualize rolling statistics like mean, median, and standard deviation to see how they evolve over time.
                        \end{itemize}
                    \item Analyze Stationarity:
                        \begin{itemize}
                            \item Check for stationarity, meaning the statistical properties of the data don't change over time. Non-stationary data can lead to inaccurate models.
                            \item Use tests like Augmented Dickey-Fuller (ADF) to assess stationarity.
                            \item Apply differencing (e.g., differencing once) to achieve stationarity if necessary.
                        \end{itemize}
                    \item Examine Autocorrelation and Partial Autocorrelation:
                        \begin{itemize}
                            \item Plot autocorrelation (ACF) and partial autocorrelation (PACF) plots to analyze the lag at which previous values influence future values.
                            \item Identify significant lags for potential feature engineering or model selection.
                        \end{itemize}
                    \item Explore Decompositions:
                        \begin{itemize}
                            \item Perform seasonal decomposition if seasonality is evident.
                            \item Analyze trend, seasonality, and residuals separately to understand underlying components.
                        \end{itemize}
                    \item Feature Engineering:
                        \begin{itemize}
                            \item Based on your insights, create new features like lagged values, rolling metrics, or seasonal indicators.
                            \item These features can capture important information and improve model performance.
                        \end{itemize}
                    \item Anomaly Detection:
                        \begin{itemize}
                            \item Identify potential anomalies or outliers that might affect your model.
                            \item Use techniques like Z-scores or isolation forests to flag unusual data points.
                        \end{itemize}
                    \item Document and Report:
                        \begin{itemize}
                            \item Document your findings and insights from the EDA in a clear and concise way.
                            \item This helps you and others understand the data characteristics and guide model selection.
                        \end{itemize}
                    \item Additional Tips:
                        \begin{itemize}
                            \item Consider domain knowledge and specific problem context while interpreting results.
                            \item Use interactive visualizations and tools for deeper exploration and insights.
                            \item Compare different time series plots and statistics to understand various aspects of the data.
                        \end{itemize}
                    \end{itemize}
                
                Remember, effective EDA is crucial for setting the stage for successful time series modeling and analysis.


